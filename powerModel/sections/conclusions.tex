
The results of our experiments are evaluated by assuming two primary use cases for a power consumption model in autonomous UAV planning and control. The first case is predicting the total power consumed by a trajectory, which requires the power consumption model to be accurate across all states in the trajectory. The second case is determining the parameters of minimal power trajectories, where the model’s gradient should have the same sign as the actual power consumed gradient with respect to controllable states. For both cases, our results indicate that a hybrid model based on $v_\infty$ may be necessary.

\subsection{Estimating total power}
As seen in Fig. 4 and Table 1, $P_{ind}$ is by far the largest contributor to accurate estimates of power consumption, especially at lower $v_g$. As expected, the contribution of $P_{par}$ increases with $v_g$, but is still dominated by $P_{ind}$. Interestingly, it is evident that the inclusion of $P_{pro}$ degrades the estimate at greater $v_g$. This could be attributed to the assumption used to neglect a term discussed in \ref{sec:Power} or the method for determining the constant $\kappa_1$. Since all our flights were conducted at near-level flight, it is expected that $P_c$ is nearly negligible.

Table 1 also shows that our full model achieves relative errors lower than previously published models. Since we did not evaluate any of these models using our data, it is not a fair comparison, but it is still worth noting until further verification. This is of particular interest to path and motion planning problems that keep a large margin of battery capacity. By having a better estimate of $P$, uncertainty around max flight time decreases, allowing for a larger feasible trajectory set.

\subsection{Determining minimal power trajectories}
For model-based methods that use gradient descent, it is ideal that the model's computed gradient's descent direction aligns with the descent direction of the underlying physics. Otherwise, the algorithm will converge to non-minimal states. To evaluate this, we approximate the gradient of \eqref{powConsumed} with respect to $v_\infty$, $\theta$, and $\phi$ by taking a finite difference between samples taken while orbiting. In Fig. 4, we see that removing $P_{ind}$ causes discrepancies between the descent directions of $\nabla P$ and $\nabla \hat{P}_{-ind}$. Fig. 4 also shows that this problem diminishes as $v_g$ increases, but as previously stated this magnitude of $v_g$ approaches the current dynamic limit of commercial platforms. Based on that, $P_{ind}$ should always be present in a model where some form of gradient descent is used.

We want to highlight the reliance of our model on accurate estimations of the 3D, time-varying wind field. We think there is room for significant improvement in solutions to this problem, particularly in the atmospheric boundary layer and urban canopies. Looking forward, it will be important to not only estimate the wind vector directly acting on the UAV, like current methods, but to predict the wind field of the entire operating area as well.
