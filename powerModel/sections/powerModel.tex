
In this section, we present the derivations for each term of the power consumption model based on models for general rotorcraft given in \cite{leishman2006principles} and models specific to quadrotors used in \cite{ware2016analysis,kreciglowa2017energy,tagliabue2019model,liu2017power}. 

Based on the general rotorcraft models, we assume that the total aerodynamic power required is the sum of induced power, parasitic power, profile power, and power to climb in altitude. The first three terms are the power required to overcome the drag force of the same name.
\begin{align}
	\label{powReq}
	P_{req} = P_{ind} + P_{par} + P_{pro} + P_c
\end{align}

\subsection{Induced power}
Simply put, all surfaces that produce lift consequently produce drag, defined as induced drag. Several sources cover this in much greater detail in general \cite{anderson2005introduction}, specific to rotorcraft \cite{leishman2006principles}, and specific to quadrotors \cite{bangura2012nonlinear}. Keeping with the existing quadrotor models, we model the induced power as \eqref{powInd}, which is a simplified model given in \cite{leishman2006principles}
\begin{align}
    \label{powInd}
    P_{ind} &= T \left(v_i + v_\infty \sin{\phi} \sin{\theta}\right)
\end{align}
where $v_i$ is the velocity induced by air flowing through the rotors and is defined implicitly by \cite{leishman2006principles} as
\begin{align}
    v_i &= \frac{v_h^2}{\sqrt{\left(v_\infty \cos{\alpha} \right)^2 + \left(v_i + v_\infty \sin{\alpha} \right)^2}}
\end{align}
where $v_h$ is the induced velocity at hover and can be found theoretically using \eqref{indHover} where $r$ is the rotor radius. This expression for $v_i$ can be manipulated in to a quartic function and solved numerically \cite{hoffmann2007quadrotor}.
\begin{align}
	\label{indHover}
	v_h = \sqrt{\frac{mg}{2 \rho \pi r^2}}
\end{align}

\subsection{Parasitic power}
Parasitic drag comes from all nonlifting surfaces and is the predominant source of drag at higher velocities. It is often considered negligible at lower velocities for quadrotors \cite{bangura2012nonlinear}. We model the power to overcome parasitic drag as \eqref{powPara}, which is also given in \cite{leishman2006principles}, and where $f_D$ is modeled by \eqref{DragEqn}.
\begin{align}
	\label{powPara}
	P_{par} = f_D v_\infty 
\end{align}

\subsection{Profile power}
Profile drag is generated by the transverse velocity of the rotors as they rotate. While hovering, it is equal to $0$ since the opposing blades of the rotor cancel each other. We use \eqref{powPro} to model the power required to overcome profile drag where $\kappa_1$ and $\kappa_2$ are determined empirically \cite{bangura2012nonlinear,liu2017power}.
\begin{align}
	\label{powPro}
	P_{pro} = \kappa_1 T^{3/2} + \kappa_2 \left(V_\infty \cos \phi \cos \theta \right)^2 T^{1/2}
\end{align}
Standard derivations of $P_{pro}$ use blade element theory, which can be seen in more detail in \cite{liu2017power}. As Liu et al. states, $\kappa_2$ is often approximately $0$, so the second term of \eqref{powPro} can be neglected.

\subsection{Climb power}
This is the only term in our model that is not associated with overcoming drag. The power necessary to climb is simply the weight of the quadrotor multiplied by the velocity in the global $z$ direction as seen in \eqref{powClimb}. 
\begin{align}
	\label{powClimb}
	P_{c} = v_z^W m g
\end{align}
Intuitively, it is the power required to change the quadrotor's potential energy.

\subsection{Efficiency}
To convert $P_{req}$ to an estimate of electrical power consumed, $\hat{P}$, we join \cite{ware2016analysis}, \cite{kreciglowa2017energy}, and \cite{tagliabue2019model} in assuming that all losses due to electrical components can be aggregated to one efficiency term $\eta$. This gives us \eqref{powConsumed}, which is referred to as the full model for the remainder of the paper.
\begin{align}
	\label{powConsumed}
    \hat{P} = \frac{1}{\eta} \left(P_{ind} + P_{par} + P_{pro} + P_{c} \right)
\end{align}
$\eta$ is defined as the ratio between the theoretical power required at hover, which is found using \eqref{indHover}, and the mean power consumed during our hover trials which are detailed in \ref{sec:Exp}.