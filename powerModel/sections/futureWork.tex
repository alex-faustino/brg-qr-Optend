
In this paper we presented an updated white box model for quadrotor UAV power consumption, which achieves relative errors below 12 $\%$ across a range of flight conditions. We evaluated the contribution of each term in the model with respect to accuracy of the power estimate and its gradient. By assuming level flight and planar wind, our model only depends on roll, pitch, and $v_\infty$. We varied these parameters by flying planar orbits at different ground velocities, $v_g$. We showed empirically that the $P_{ind}$ term is the greatest contributor to reducing total error across all flight regimes we studied. Agreeing with previous work we also showed that $P_{par}$ contributes more to error reduction as $v_g$ increases. To reduce error at larger velocities, $P_{pro}$ should be ignored; this suggests the use of a hybrid model. Additionally, with respect to the power gradient having the correct sign, $P_{ind}$ is extremely important at lower $v_g$, but is nearly negligible at larger velocities. 

Moving forward, a hybrid or weighted model should be designed and validated using the data from these experiments. This new model could then be utilized as the power consumption model for power minimization methods in experiment and practice. Based on the findings of experiments in this paper, incorporating that model should reduce power consumption relative to previous implementations. Additionally, we want to focus on improving the wind estimation techniques that our model relies heavily on. Particularly, moving towards onboard estimation of low altitude and urban canopy wind fields at large. Improving methods for wind estimation will be beneficial to UAV planning problems in general not just quadrotors.
