
The last decade has seen quadrotor helicopters explode in popularity. From an emerging unmanned aerial vehicle (UAV) concept to a prominent research and commercial platform \cite{kumar2012opportunities,hoffmann2007quadrotor} quadrotors have become the nearly ubiquitous aerial robot. Their relative low cost and the simplicity of their dynamics when near hover \cite{bouabdallah2004pid} has made them popular in numerous applications \cite{heng2015efficient,roberts2017submodular,frazzoli2002real}. 

No matter the application, if the quadrotor is autonomous, all its motion is occurring near hover. This condition introduces the quadrotor platform's greatest weakness, power efficiency. In \cite{kumar2012opportunities}, Kumar et al. state that the power required for a quadrotor to maintain hover is approximately 200 W per kg. Additionally, due to current LiPo battery technology a quadrotor's battery can be 25-30\% of its total mass. These two purely hardware constraints restrict the size of an area that can be explored, the number of images that can be captured by an onboard camera, the mass of potential payloads, etc.. The problem of quadrotor power efficiency therefore creates a practical limit on the platform's utility. 

Currently, most approaches to solving this problem can be classified in to three categories, listed in descending order of prevalence: 
\begin{enumerate}
	\item hardware focused optimization
	\item algorithm (software) focused optimization
	\item development of bio-inspired, hybrid systems
\end{enumerate}
Traditionally, the bulk of work done to increase quadrotor efficiency is in the first category. Efforts here consist mostly of reducing the weight of materials such as the airframe, sensors, and power electronics. The second category has grown in popularity in recent years due to the increased capabilities of onboard computers. The most common methods here involve incorporating vehicle power consumption in to cost functions of existing optimal planning and control algorithms. Finally, the third category often produces novel systems that increase efficiency by tranisitioning to another dynamic mode such as perching, walking, or rolling. In \cite{karydis2017energetics}, Karydis et al. provide a thorough review of some of the most promising work in all three categories. Category one and three are not addressed further in this paper as the scope falls entirely in category two.

In this paper we present a solution for finding power optimal quadrotor trajectories between two configurations in SE(3) that minimize the total amount of power required, $P_{tot}$, to complete. Our solution differs from current methods by allowing for a continuous action space and removing fixed altitude constraints present in current methods. The remainder of the paper is organized as follows: Section II broadly reviews similar approaches already in the literature; Section III describes the dynamic model of the quadrotor including power consumption; Section IV details the formulation of our optimal control problem; Section V presents the results from our experiments in simulation; and Section VI contains discussion of future work and conclusions from this work.

\begin{figure}[t]
    \label{QuadAction}
	\centering
	\includegraphics[width=0.5\textwidth, trim={10cm 6cm 10cm 5cm},clip]{quad-action.pdf}
	\caption{Example of a quadrotor flying with a constant forward velocity where the pitch angle is small enough that the dynamics about each axis are decoupled when linearized.}
\end{figure}
