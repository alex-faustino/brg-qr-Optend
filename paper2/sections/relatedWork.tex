
Several promising approaches for increasing small UAV endurance through software have been made in the last four years. Di Franco and Buttazzo present one of the earliest successful methods in \cite{di2015energy} where they use energy as an additional optimization criteria for coverage problems. They fit energy models to empirical data collected from four basic maneuvers: take-off, straight and level flight, level turn, and landing. Their algorithm then finds a set of nodes at a fixed altitude that maximizes coverage with a given resolution. The edges of the graph are one of the four basic maneuvers they have an energy model for. An optimization scheme then finds the constant velocity that minimizes the energy consumed by each edge.

The major drawback of this approach is that the UAV must come to a stop at each node consuming a large amount of energy while accelerating and decelerating. Our approach alleviates this issue by allowing for smooth trajectories.

One of the most promising approaches comes from Morbidi et. al \cite{morbidi2016minimum} where they leverage a brushless DC motor model to find minimum energy paths with respect to the angular acceleration of the rotors in continuous time. Their approach has relatively lax constraints, the initial and final angular acceleration of the rotors must be equal, and has a polynomial objective function. 
\begin{align*}
    E &= \int\limits_{t_i}^{t_f} \sum\limits_{j=1}^{4} (c_0 + c_1 \omega_j(t) + c_2 \omega_j(t)^2 \\
    &\quad + c_3 \omega_j(t)^3 + c_4 \omega_j(t)^4 + c_5 \dot{\omega}_j(t)^2) \text{ }dt
\end{align*}
With a clever change of variables the optimization problem becomes simple enough that it can be solved numerically. An issue with this method is that it has only been validated in simulation and there is currently no method for determining the polynomial coefficients of the objective function for a real system. While our method is also only currently validated in simulation it only relies on constants that have established empirical methods for determining them.

Taking a step up in complexity are approaches more similar to ours that aim to greater increase endurance gains by including aerodynamic effect considerations. Ware and Roy \cite{ware2016analysis} incorporate urban wind data to find more efficient trajectories between two points in the urban canopy layer. Using the wind's prevailing speed and direction above the surrounding buildings as the input to a CFD solver they generate a grid with 1 m resolution. Similar to \cite{di2015energy} they create a graph for a fixed altitude such that each interior node has eight edges. Using the wind vector, $v_w$, from the CFD solution they can then choose an upper and lower bounded ground velocity, $v_g$, for the quadrotor such that it minimizes:
\begin{align*}
    E_i = \frac{T(v_i + v_\infty \sin{\alpha}) (v_g - v_w) \|d\|}{v_g}
\end{align*}
where $\|d\|$ is the Euclidean distance between the two nodes. They address the acceleration problem encountered by \cite{di2015energy} by constraining the change in velocity, $\Delta v_g$, between edges. Their simulation results show that wind aware planning uses less power than wind naive planning and highlights the importance of having an estimate of the local wind field. In our work we also assume that the local wind field is known, however we focus on a general, time varying wind field (more similar to \cite{kai2017nonlinear}) rather than a domain specific solution.

Most recently, Tagliabue et. al \cite{tagliabue2019model} presented a model-free control approach that uses an extremum seeking controller t0 converge to the velocity that minimizes energy consumption on a given trajectory. The extremum seeking controller takes the voltage and current draw measured at the battery terminals as inputs and ouputs a velocity for the flight controller to track. This approach requires little knowledge about your system and environment which they demonstrate by attaching packages to the quadrotor completely altering its dynamics and aerodynamics. Theirs is one of the only methods to be verified empirically on a physical system, however the experiments did not include wind disturbance. A potential downside to this approach is that it minimizes energy consumed on a predetermined trajectory rather than determining a minimal energy trajectory. For practical applications, combining this method with a method similar to ours would likely produce better results. 


