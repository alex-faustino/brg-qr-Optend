 
The last decade has seen quadrotor helicopters explode in popularity. From an emerging unmanned aerial vehicle (UAV) concept to a prominent research and commercial platform \cite{kumar2012opportunities,hoffmann2007quadrotor} quadrotors have become the nearly ubiquitous aerial robot. Their relative low cost and the simplicity of their dynamics when near hover \cite{bouabdallah2004pid} has made them popular in numerous applications \cite{heng2015efficient,roberts2017submodular,frazzoli2002real}.

The condition of operating near hover also introduces the quadrotor platform's greatest weakness, power efficiency. The power required to keep a quadrotor at hover is approximately 200 W per kg \cite{kumar2012opportunities}. Since quadrotors are rarely flown dynamically far away from hover when used in application, the problem of power efficiency creates a practical limit on their utility. This limit restricts the size of an area that can be explored, the number of images that can be captured by a camera, the mass of potential payloads, etc.. In robotics, maximizing endurance is mostly thought of as a design problem handled by manufacturers. In this paper we present evidence that choices about trajectory and velocity also have a meaningful effect on flight time.

Determining an aerial vehicle's endurance is a common problem in flight mechanics. Solutions for fixed wing and rotor aircraft maximum endurance in steady, forward flight are well known and widely used \cite{anderson2005introduction,leishman2006principles}. Maximum endurance is achieved by traveling at the relative velocity, $V_\infty$, where the power required, $P_{req}$, to overcome the drag force, $D$, is at a minimum. We call this velocity $V_{me}$. 


\begin{figure}[ht]
    \label{QuadDiagram}
	\centering
	\includegraphics[width=8cm]{placeholder-image.jpg}
	\caption{Coordinate system and flow diagram illustrating induced velocity, force balance at equilibrium, etc..}
\end{figure}
